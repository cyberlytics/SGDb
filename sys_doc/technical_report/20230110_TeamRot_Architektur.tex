\documentclass[a4paper, 10pt, conference]{IEEEtran}

\usepackage[utf8]{inputenc}
\usepackage[T1]{fontenc}
\usepackage[
	colorlinks=true,
	linkcolor=black,
	anchorcolor=black,
	citecolor=black,
	filecolor=black,
	menucolor=black,
	runcolor=black,
	urlcolor=black
]{hyperref}
\usepackage{url}
\usepackage{graphicx}
\usepackage[ngerman]{babel}
\usepackage[style=ieee]{biblatex}
\usepackage{rotating}

\addbibresource{references.bib}

\graphicspath{ {./images/} }

\title{\LARGE
\textbf{SGDb: Semantic Video Game Database} \\ Technical Report
}

\author{
\IEEEauthorblockN{Anastasia Chernysheva} \IEEEauthorblockA{\textit{a.chernysheva@oth-aw.de}}\and
\IEEEauthorblockN{Jakob Götz} \IEEEauthorblockA{\textit{j.goetz@oth-aw.de}}\and
\IEEEauthorblockN{Ardian Imeraj} \IEEEauthorblockA{\textit{a.imeraj@oth-aw.de}}\and
\IEEEauthorblockN{Patrice Korinth} \IEEEauthorblockA{\textit{p.korinth@oth-aw.de}}\and
\IEEEauthorblockN{Philipp Stangl} \IEEEauthorblockA{\textit{p.stangl1@oth-aw.de}}\and
}

\begin{document}

\maketitle
\thispagestyle{empty}
\pagestyle{empty}

\begin{abstract}
Dieser Technical Report beschreibt die Architektur von SGDb -- eine webbasierte Anwendung mit einer Graphen-basierten Suche von Videospielen.
\end{abstract}

\section{Einführung und Ziele}

In den weiteren Abschnitten des Technical Reports wird zuerst auf die Lösungsstrategie in Abschnitt~\ref{sec:loesungsstrategie} eingegangen.
Im Nächsten Abschnitt~\ref{sec:bausteinsicht} wird das Gesamtsystem aus Bausteinsicht beschrieben.
Anschließend wird in Abschnitt~\ref{sec:verteilungssicht} die Verteilungssicht der Anwendung beschrieben.
In Abschnitt~\ref{sec:entwicklungswerkzeuge} werden die angewandten Werkzeuge zur Entwicklung der Anwendung vorgestellt.
Abschließend wird ein Fazit und Ausblick in Abschnitt~\ref{sec:fazit} gegeben.


\section{Lösungsstrategie}\label{sec:loesungsstrategie}

Dieser Abschnitt enthält einen stark verdichteten Architekturüberblick.
Eine Gegenüberstellung der wichtigsten Ziele und Lösungsansätze.


\section{Bausteinsicht}\label{sec:bausteinsicht}
Diese Sicht zeigt die statische Zerlegung des Systems in Bausteine sowie deren Beziehungen.

\subsection{Gesamtsystem}\label{subsec:gesamtsystem}


\subsection{Backend}\label{subsec:backend}
Für die Datenbeschaffung wird eine Client-Anfrage an die Endpunkte der IGDB-API gesendet. 
Als Antwort wird im Anschluss der Bearbeitung ein Datensatz zurückgegeben, der wesentliche Informationen zu einem Spiel 
beinhaltet. Zu den Informationen zählen: Spiele-ID, Spieletitel, Beschreibung, Veröffentlichungsdatum, Genre, 
Spieleplattform, Bewertung, Entwicklername- und Land, sowie eine URL des Coverbildes. Der Datensatz wird im 
JSON-Format gespeichert und auf 500 Spiele reduziert, die eine hohe Bewertung erzielt haben (>85). Begründet wird diese 
Bedingung durch das Bestreben einen Datensatz zu erzeugen, der über viele Informationen zu den einzelnen Spielen verfügt. 
Der Datensatz weist an mancher Stelle numerische Werte auf, die jedoch nicht informativ wären für einen Durschnittsnutzer. 
So ist das Veröffentlichungsdatum im Unix-Zeitstempel angegeben. Dieser ist auch als "The Epoch" bekannt und zählt die 
Anzahl der vergangenen Sekunden seit 1.Januar 1970 \cite{unix}. Der Standort des Entwicklerunternehmens wird als 
dreistelliger Country-Code definiert, der von der International Organization for Standardization (ISO) entwickelt 
und im ISO-3166 publiziert wurde \cite{iso}. Es wurde eine Funktion implementiert, die die Unix-Zeit in ein reguläres 
Zeitformat \textit{'DD-MM-YYYY'} und den ISO-Country-Code in Ländernamen konvertiert. 

Der Datensatz wird anschließend mit Refine \cite{refine} strukturiert und in RDF-Triples (Graphs) umgewandelt. 
Für die Datenverwaltung und Vernetzung der Informationen wird die Ontotext GraphDB \cite{graphdb} verwendet.  

\subsection{Frontend}\label{subsec:frontend}
Das Frontend ist für die Benutzerschnittstelle verantwortlich.
Hauptbestandteile des Frontends sind die Suchmaske, der Graph und die Detailseite zu einem Videospiel.
Für die Darstellung des Graphen und die Interaktion mit diesem wird Sigma.js~\cite{sigma} verwendet.
Sigma.js rendert Graphen mit WebGL. Damit lassen sich größere Graphen schneller zeichnen als mit Canvas- oder SVG-basierten Lösungen.
Das Graphenmodell wird in einer separaten Bibliothek namens Graphology~\cite{graphology} verwaltet.
Dies ist eine Standardbibliothek mit Algorithmen aus der Graphentheorie und allgemeinen Hilfsprogrammen wie z.B.\ Graphengeneratoren.

Generierung
Für das Graphen Layout wird der ForceAtlas2 Algorithmus~\cite{forceatlas2} verwendet.
Vor dem Start von ForceAtlas 2 Layout muss die Startposition jedes Knotens festgelegt werden.
Daher müssen zwei Attribute namens x und y für alle Knoten des Diagramms definiert werden.

Dazu wird ein Graph Objekt mit dem Random layout erzeugt.
Positionierung jedes Knotens, indem die Koordinaten gleichmäßig nach dem Zufallsprinzip auf dem Intervall [0, 1) auswählt. %(https://graphology.github.io/standard-library/layout.html)

Die Interaktion mit dem Backend erfolgt über die REST-Schnittstelle.


\section{Verteilungssicht}\label{sec:verteilungssicht}


\section{Entwicklungswerkzeuge}\label{sec:entwicklungswerkzeuge}


\section{Fazit und Ausblick}\label{sec:fazit}

Im Rahmen der Arbeit wurde mit einem begrenzten Datensatz von 500 Spielen eine erste Version der Anwendung entwickelt.


\printbibliography

\end{document}

