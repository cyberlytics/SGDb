\documentclass[a4paper, 10pt, conference]{IEEEtran}

\usepackage[utf8]{inputenc}
\usepackage[T1]{fontenc}
\usepackage[
	colorlinks=true,
	linkcolor=black,
	anchorcolor=black,
	citecolor=black,
	filecolor=black,
	menucolor=black,
	runcolor=black,
	urlcolor=black
]{hyperref}
\usepackage{url}
\usepackage{graphicx}
\usepackage[ngerman]{babel}
\usepackage[style=ieee]{biblatex}
\usepackage{rotating}

\addbibresource{references.bib}

\graphicspath{ {./images/} }

\title{\LARGE
\textbf{SGDb: Semantic Video Game Database} \\ Technical Report
}

\author{
\IEEEauthorblockN{Anastasia Chernysheva} \IEEEauthorblockA{\textit{a.chernysheva@oth-aw.de}}\and
\IEEEauthorblockN{Jakob Götz} \IEEEauthorblockA{\textit{j.goetz@oth-aw.de}}\and
\IEEEauthorblockN{Ardian Imeraj} \IEEEauthorblockA{\textit{a.imeraj@oth-aw.de}}\and
\IEEEauthorblockN{Patrice Korinth} \IEEEauthorblockA{\textit{p.korinth@oth-aw.de}}\and
\IEEEauthorblockN{Philipp Stangl} \IEEEauthorblockA{\textit{p.stangl1@oth-aw.de}}\and
}

\begin{document}

\maketitle
\thispagestyle{empty}
\pagestyle{empty}

\begin{abstract}
Dieser Technical Report beschreibt die Architektur von SGDb -- eine webbasierte Anwendung mit einer Graphen-basierten Suche von Videospielen.
\end{abstract}

\section{Einführung und Ziele}

In den weiteren Abschnitten des Technical Reports wird zuerst auf die Lösungsstrategie in Abschnitt~\ref{sec:loesungsstrategie} eingegangen.
Im Nächsten Abschnitt~\ref{sec:bausteinsicht} wird das Gesamtsystem aus Bausteinsicht beschrieben.
Anschließend wird in Abschnitt~\ref{sec:verteilungssicht} die Verteilungssicht der Anwendung beschrieben.
In Abschnitt~\ref{sec:entwicklungswerkzeuge} werden die angewandten Werkzeuge zur Entwicklung der Anwendung vorgestellt.
Abschließend wird ein Fazit und Ausblick in Abschnitt~\ref{sec:fazit} gegeben.


\section{Lösungsstrategie}\label{sec:loesungsstrategie}

Dieser Abschnitt enthält einen stark verdichteten Architekturüberblick.
Eine Gegenüberstellung der wichtigsten Ziele und Lösungsansätze.


\section{Bausteinsicht}\label{sec:bausteinsicht}
Diese Sicht zeigt die statische Zerlegung des Systems in Bausteine sowie deren Beziehungen.

\subsection{Gesamtsystem}\label{subsec:gesamtsystem}


\subsection{Backend}\label{subsec:backend}


\subsection{Frontend}\label{sub:frontend}


\section{Verteilungssicht}\label{sec:verteilungssicht}


\section{Entwicklungswerkzeuge}\label{sec:entwicklungswerkzeuge}


\section{Fazit und Ausblick}\label{sec:fazit}

Im Rahmen der Arbeit wurde mit einem begrenzten Datensatz von 500 Spielen eine erste Version der Anwendung entwickelt.


\printbibliography

\end{document}

