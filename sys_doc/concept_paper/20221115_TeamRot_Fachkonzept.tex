\documentclass[a4paper, 10pt, conference]{IEEEtran}

\IEEEoverridecommandlockouts

\usepackage[utf8]{inputenc}
\usepackage[T1]{fontenc}
\usepackage[colorlinks=true,linkcolor=black,anchorcolor=black,citecolor=black,filecolor=black,menucolor=black,runcolor=black,urlcolor=black]{hyperref}
\usepackage{graphicx}
\usepackage[ngerman]{babel}
\usepackage[style=ieee]{biblatex}

\addbibresource{references.bib}

\graphicspath{ {./images/} }

\begin{document}

\title{\LARGE \bf
Titel
}

\author{
\IEEEauthorblockN{Anastasia Chernysheva} \IEEEauthorblockA{\textit{a.chernysheva@oth-aw.de}}\and
\IEEEauthorblockN{Jakob Götz} \IEEEauthorblockA{\textit{j.goetz@oth-aw.de}}\and
\IEEEauthorblockN{Ardian Imeraj} \IEEEauthorblockA{\textit{a.imeraj@oth-aw.de}}\and
\IEEEauthorblockN{Patrice Korinth} \IEEEauthorblockA{\textit{p.korinth@oth-aw.de}}\and
\IEEEauthorblockN{Philipp Stangl} \IEEEauthorblockA{\textit{p.stangl1@oth-aw.de}}\and
}

\maketitle
\thispagestyle{empty}
\pagestyle{empty}

\begin{abstract}
TODO
\end{abstract}

\section{Einleitung}

TODO


\section{Verwandte Arbeiten} \label{s:verwandte_arbeiten}

TODO


\section{Anforderungen} \label{s:anforderungen}

In der Anforderungsanalyse wurden zwei Stakeholder identifiziert: Benutzer und Entwickler. Deren Anforderungen werden in diesem Abschnitt in Form von User Stories beschrieben.

\subsection{Anforderung}

Als (\dots) möchte ich (\dots), damit (\dots). Akzeptanzkriterien sind:
\begin{itemize}
\item Item 1
\item Item 2
\item Item 3
\end{itemize}

\subsection{(Optional) Anforderung}

Als (\dots) möchte ich (\dots), damit (\dots). Akzeptanzkriterien sind:
\begin{itemize}
\item Item 1
\item Item 2
\item Item 3
\end{itemize}

\section{Methoden} \label{s:methoden}

Ziel bei der Entwicklung und bei der späteren Verwendung ist eine platformunbhängige Anwendung, die eine Schnittstelle für die \textit{Linked Open Data Cloud} als auch der Bedienoberfläche der Anwendung bereitstellt. Daraus ergeben sich folgende technische Schlüsselbausteine:

Für die Repräsentation der semantischen Daten im Frontend wird das Framework Svelte verwendet. Die Daten werden voraussichtlich in einer Ontotext GraphDB Instanz persistiert. Im Backend wird FastAPI für eine REST-Schnittstelle und RDFLib zum arbeiten mit RDF verwendet. Die Kommunikation zwischen Frontend und Backend wird über eine RESTful-API abgewickelt. Um eine fehlerfreie Anwendung zu entwickeln werden geeignete Test-Frameworks für Frontend als auch Backend selektiert und verwendet. Die einzelnen Komponenten der Anwendung werden mittels Docker \textit{containerisiert}.


\printbibliography

\end{document}
