\documentclass[a4paper, 10pt, conference]{IEEEtran}

\IEEEoverridecommandlockouts

\usepackage[utf8]{inputenc}
\usepackage[T1]{fontenc}
\usepackage[colorlinks=true,linkcolor=black,anchorcolor=black,citecolor=black,filecolor=black,menucolor=black,runcolor=black,urlcolor=black]{hyperref}
\usepackage{graphicx}
\usepackage[ngerman]{babel}
\usepackage[style=ieee]{biblatex}

\addbibresource{references.bib}

\graphicspath{ {./images/} }

\begin{document}

\title{\LARGE \bf
Semantic Video Game Database 
}

\author{
\IEEEauthorblockN{Anastasia Chernysheva} \IEEEauthorblockA{\textit{a.chernysheva@oth-aw.de}}\and
\IEEEauthorblockN{Jakob Götz} \IEEEauthorblockA{\textit{j.goetz@oth-aw.de}}\and
\IEEEauthorblockN{Ardian Imeraj} \IEEEauthorblockA{\textit{a.imeraj@oth-aw.de}}\and
\IEEEauthorblockN{Patrice Korinth} \IEEEauthorblockA{\textit{p.korinth@oth-aw.de}}\and
\IEEEauthorblockN{Philipp Stangl} \IEEEauthorblockA{\textit{p.stangl1@oth-aw.de}}\and
}

\maketitle
\thispagestyle{empty}
\pagestyle{empty}

\section{Einleitung}

Das erste Videospiel \textit{Tennis for Two} aus dem Jahr 1958
ebnete den Weg für die Spieleindustrie. Über Jahrzehnte
hinaus kamen immer mehr verschiedene Spiele auf den Markt,
die sich in ihrer Art und Weise unterscheiden. Mit den
heutigen Videospielen ist es möglich, sich mit Spielern aus der
ganzen Welt zu messen oder gemeinsam zu spielen. Aus den
unzählig veröffentlichen Videospielen ist eine breite Palette an
Genres entstanden. Dabei wird in der Spieleindustrie grob in
sechs Kategorien unterschieden, hierbei zählen die Strategieund
Action-Spiele zu den beliebtesten Genres \cite{statistica}. Zu den
anderen vier Kategorien zählen hier die Rollen-, Adventure-
, Simulationen- und Sonstigespiele. Jedes dieser Kategorien
besitzt weitere Unterkategorien, die sich in ihrer Art und Weise
unterscheiden.

Um einen Überblick zu erhalten, wie die Spiele zusammenhängen, präsentieren wir hier eine plattformunabhängige Anwendung, welche die Spiele in einem Graphen darstellt. In Abschnitt~\ref{s:verwandte_arbeiten} des Konzeptpapiers werden bereits bestehende Anwendungen vorgestellt, die sich mit der Kategorisierung von Spielen und Darstellung von Graphen beschäftigen. In Abschnitt~\ref{s:anforderungen} werden die Anforderungen in Form von User Stories beschrieben. Zum Schluss wird auf die zu verwendenden Technologien zur Umsetzung des Projekts eingegangen, welche in Abschnitt~\ref{s:methoden} näher beschrieben werden.



\section{Verwandte Arbeiten} \label{s:verwandte_arbeiten}
Die Anwendung Obsidian erstellt eine Graphenoberfläche
aus von Nutzern kreierten Markdown Dateien. Über Verlinkungen
zwischen den Markdown Dateien wächst der Graph
an. Der Nutzer kann den Graph filtern oder gewisse Teile daraus
hervorheben. Diese Anwendung zeigt hervorragend, wie
eine Visualisierung eines Graphen möglicherweise aussehen
könnte.

Die Website igdb.com gibt einen Eindruck, was der Nutzer
von einer Spielesuche erwarten kann. Der Funktionsumfang
und die Spieleanzahl ist groß, sodass das Suchergebnis meist
zufriedenstellend ist. Des Weiteren besitzt die Website eine
gut dokumentierte Entwickler-API. Dies ermöglicht es Entwicklern,
Daten in Form von JSON, REST oder ProtoBuf
zu erhalten. Jedoch sind die Daten, welche IGDB bereitstellt
nicht für die LOD-Cloud aufbereitet. Außerdem ist die Oberfläche
der Website sehr klassisch aufgebaut, wodurch kein
weiteres Interesse geweckt wird. Hier setzt unser Projekt an.
Es wird eine Web-Applikation entworfen, welche die Daten
anreichert, damit diese in die LOD-Cloud überführt werden
können. Zusätzlich wird eine Graphenoberfläche entworfen, die dazu einlädt, artverwandte Spiele auch ohne Filteroptionen zu entdecken.

\section{Anforderungen} \label{s:anforderungen}
% Gehören sich Entwickler in "User-Stories" mit rein? haben ja nur US: Testabdeckung für Entwickler?
In der Anforderungsanalyse wurden zwei Stakeholder identifiziert: Benutzer und Entwickler. Deren Anforderungen werden in diesem Abschnitt in Form von User Stories beschrieben.

% Laut Prof. Neumann Folien sollen wir UI nicht in User-Stories erwähnen. Aber bei uns ist das ja ein zentraler Punkt?
\subsection{Graphenoberfläche}
Als Benutzer möchte ich eine Graphenoberfläche, die visuell
Verbindungen zu anderen Spielen aufzeigt. Somit kann ich
direkt ähnliche Spiele, die mir ebenfalls gefallen könnten,
sehen. Akzeptanzkriterien sind:
\begin{itemize}
\item Graphenoberfläche soll direkt sichtbar sein beim Besuchen
der Website
\item Der Graph soll mit ausreichend vielen Knoten angereichert
sein
\item Die Oberfläche soll visuell ansprechend sein und zum
Stöbern einladen
\end{itemize}

\subsection{Knotendetails}
Als Benutzer möchte ich mit einem Klick auf einen Knoten
im Graphen weitere Details zu dem im Knoten enthaltenen
Spiel sehen. Somit erhalte ich ohne die Graphenoberfläche
zu verlassen die wichtigsten Informationen zu einem Spiel.
Akzeptanzkriterien sind:
\begin{itemize}
\item Knoten im Graphen sollen anklickbar sein
\item Der geklickte Knoten öffnet ein Feld mit weiteren Informationen
zu den ausgewählten Knoten
\item Die wichtigsten Informationen befinden sich zusammengefasst
im geöffneten Feld
\end{itemize}

\subsection{Filter}
Als Benutzer möchte ich mit einer Filteroption Knoten und
somit Spiele im Graphen visuell hervorheben. Dadurch kann
auch in einem großen Graphen der Nutzer eine Gruppe von
Spielen problemlos finden. Akzeptanzkriterien sind:
\begin{itemize}
\item Diverse Filtereinstellungen
\item Ergebnis sollte gut erkennbar hervorgehoben werden
\item Ausreichend viele Filteroptionen
\end{itemize}

% Die Twitter-Card User-Storie komplett umschreiben
\subsection{Twitter-Cards für das jeweilige Spiel}
Als Benutzer möchte ich Twitter-Cards zu dem Spiel sehen, welches aktuell betrachtet wird. Akzeptanzkriterien sind:
\begin{itemize}
\item Aktuelle Twitter-Cards geordnet anzeigen
\item Twitter-Cards fügen sich formschön in die Website-Oberfläche ein
\item Es existiert eine Verlinkung zum jeweiligen Twitter-Beitrag
\end{itemize}

\subsection{(Optional) Detailseite}
Als Benutzer möchte ich eine Detailseite zu einem jeweiligen
Spiel, welche über den Knoten im Graphen erreicht wird.
Somit muss ich als Nutzer nicht externe Internetseiten für
weitere Informationen besuchen. Akzeptanzkriterien sind:
\begin{itemize}
\item Alle Details eines Spiels werden aufgelistet
\item Das Spiel wird vorgestellt
\end{itemize}

\subsection{(Optional) Suche}
Als Benutzer möchte ich zusätzlich die Option haben ein
Spiel suchen zu können, da sich die Suche eines speziellen
Spiels mithilfe eines großen Graphens schwieriger gestaltet.
Akzeptanzkriterien sind:
\begin{itemize}
\item Eingabefeld für Suche
\item Weiterleitung direkt zum Knotenpunkt, der das gewünschte Spiel beinhaltet
\item Grafische Hervorhebung des Knotenpunkts
\end{itemize}

\subsection{Testabdeckung}
Als Entwickler möchte ich eine ausreichend hohe Testabdeckung, um Fehler frühzeit zu erkennen. Akzeptanzkriterien sind:
\begin{itemize}
\item Testabdeckung von mindestens 50%
\item Sowohl Frontend als auch Backend sind mit den Tests abgedeckt
\end{itemize}

\subsection{(Optional) API-Nutzung}
Als Entwickler möchte ich erkennen können, in welchen Umfang meine API genutzt wird, damit ich Rückschlüsse auf die Auslastung und Interessen erhalte. Akzeptanzkriterien sind:
\begin{itemize}
\item Wie häufig wird die API angesprochen
\item Welche API-Endpunkte werden am häufigsten genutzt
\end{itemize}

\subsection{(Optional) Einsehen meist aufgerufener Spiele}
Als Benutzer möchte ich, dass mir die meist aufgerufenen Spiele angezeigt werden, damit ich wissen kann, welche Spiele in der Spielebranche momentan gefragt sind.
Akzeptanzkriterien sind:
\begin{itemize}
\item Ein eigener Bereich für zuletzt aufgerufene Spiele auf der Startseite
\item Spiele werden jeweils nur einmal angezeigt 
\item Hierarchische Anordnung nach Anzahl der Aufrufe in bestimmter Zeit
\end{itemize}

\subsection{(Optional) Einsehen von Bewertungen}
Als Benutzer möchte ich Bewertungen zu dem jeweiligen Spiel einsehen können, damit ich mir ein Bild über die Qualität des Spiels verschaffen kann. Akzeptanzkriterien sind:
\begin{itemize}
\item Eine Anzeige in Form von Sternen oder eines Prozentsatzes
\item Eine Möglichkeit Spiele anhand der Bewertungen zu sortieren
\end{itemize}


\section{Methoden} \label{s:methoden}

Ziel bei der Entwicklung und bei der späteren Verwendung ist eine plattformunabhängige Anwendung, die eine Schnittstelle für die \textit{Linked Open Data Cloud} als auch der Bedienoberfläche der Anwendung bereitstellt. Daraus ergeben sich folgende technische Schlüsselbausteine:

Für die Repräsentation der semantischen Daten im Frontend wird das Framework Svelte verwendet. Die Daten werden voraussichtlich in einer Ontotext GraphDB Instanz persistiert. Im Backend wird FastAPI für eine REST-Schnittstelle und RDFLib zum arbeiten mit RDF verwendet. Die Kommunikation zwischen Frontend und Backend wird über eine RESTful-API abgewickelt. Um eine fehlerfreie Anwendung zu entwickeln, werden geeignete Test-Frameworks für Frontend als auch Backend selektiert und verwendet. Die einzelnen Komponenten der Anwendung werden mittels Docker \textit{containerisiert}.


\printbibliography

\end{document}
