\documentclass[a4paper, 10pt, conference]{IEEEtran}

\IEEEoverridecommandlockouts

\usepackage[utf8]{inputenc}
\usepackage[T1]{fontenc}
\usepackage[colorlinks=true,linkcolor=black,anchorcolor=black,citecolor=black,filecolor=black,menucolor=black,runcolor=black,urlcolor=black]{hyperref}
\usepackage{graphicx}
\usepackage[ngerman]{babel}
\usepackage[style=ieee]{biblatex}

\addbibresource{references.bib}

\graphicspath{ {./images/} }

\begin{document}

\title{\LARGE \bf
Titel
}

\author{
\IEEEauthorblockN{Anastasia Chernysheva} \IEEEauthorblockA{\textit{a.chernysheva@oth-aw.de}}\and
\IEEEauthorblockN{Jakob Götz} \IEEEauthorblockA{\textit{j.goetz@oth-aw.de}}\and
\IEEEauthorblockN{Ardian Imeraj} \IEEEauthorblockA{\textit{a.imeraj@oth-aw.de}}\and
\IEEEauthorblockN{Patrice Korinth} \IEEEauthorblockA{\textit{p.korinth@oth-aw.de}}\and
\IEEEauthorblockN{Philipp Stangl} \IEEEauthorblockA{\textit{p.stangl1@oth-aw.de}}\and
}

\maketitle
\thispagestyle{empty}
\pagestyle{empty}

\begin{abstract}
TODO
\end{abstract}

\section{Einleitung}

Das erste Videospiel \textit{Tennis for Two} aus dem Jahr 1958 ebnete den Weg für die Spieleindustrie. Über Jahrzehnten hinaus kamen immer mehr verschiedene Spiele auf den Markt, die sich in ihrer Art und Weise unterscheiden. Mit den heutigen Videospielen ist es möglich sich mit Spielern aus der ganzen Welt zu messen oder gemeinsam zu spielen. Aus den unzählig veröffentlichen Videospielen ist eine breite Palette an Genres entstanden. Dabei wird in der Spieleindustrie grob in sechs Kategorien unterschieden, hierbei zählen die Strategie- und Action-Spiele zu den beliebtesten Genres \cite{statistica}. Zu den anderen vier Kategorien zählen hier die Rollen-, Adventure-, Simulationen- und Sonstigespiele. Jedes dieser Kategorien besitzt weitere Unterkategorien, die sich in ihrer Art und Weise unterscheiden. 

Um einen besseren Überblick zu erhalten, präsentieren wir hier eine plattformunabhängige Anwendung, welche die Spiele in den jeweiligen Genres in eine übersichtliche  struktur zusammenfasst. In Abschnitt~\ref{s:verwandte_arbeiten} des Konzeptpapiers, werden bereits bestehende Anwendungen vorgestellt, die sich mit der Kategorisierung von Spielen beschäftigen. In Abschnitt~\ref{s:anforderungen} beschreibt die Anforderungen, in Form von User Stories. Zum Schluss wird auf die Methoden eingegangen, welche in Abschnitt~\ref{s:methoden} näher beschrieben werden.



\section{Verwandte Arbeiten} \label{s:verwandte_arbeiten}

TODO


\section{Anforderungen} \label{s:anforderungen}

In der Anforderungsanalyse wurden zwei Stakeholder identifiziert: Benutzer und Entwickler. Deren Anforderungen werden in diesem Abschnitt in Form von User Stories beschrieben.

\subsection{Selection von Genre}
Als Benutzer möchte ich mir ein Überblick über die unterschiedlichen Genres verschaffen. Akzeptanzkriterien sind:
\begin{itemize}
\item Button zum Auswählen der Kategorie
\item Unterkategorien werden in einer Liste angezeigt
\end{itemize}

\subsection{Auflistung von Spielen}
Als Benutzer möchte ich meine Genre-Favoriten eingeben und einen Vorschlag an Spielen erhalten. Akzeptanzkriterien sind:
\begin{itemize}
\item Eingabefeld für Favoriten
\item Spiele absteigen in einer Liste anzeigen
\item Button zum Suchen der Spiele
\end{itemize}

\subsection{(Optional) Anforderung}

Als (\dots) möchte ich (\dots), damit (\dots). Akzeptanzkriterien sind:
\begin{itemize}
\item Item 1
\item Item 2
\item Item 3
\end{itemize}

\section{Methoden} \label{s:methoden}

Ziel bei der Entwicklung und bei der späteren Verwendung ist eine platformunbhängige Anwendung, die eine Schnittstelle für die \textit{Linked Open Data Cloud} als auch der Bedienoberfläche der Anwendung bereitstellt. Daraus ergeben sich folgende technische Schlüsselbausteine:

Für die Repräsentation der semantischen Daten im Frontend wird das Framework Svelte verwendet. Die Daten werden voraussichtlich in einer Ontotext GraphDB Instanz persistiert. Im Backend wird FastAPI für eine REST-Schnittstelle und RDFLib zum arbeiten mit RDF verwendet. Die Kommunikation zwischen Frontend und Backend wird über eine RESTful-API abgewickelt. Um eine fehlerfreie Anwendung zu entwickeln werden geeignete Test-Frameworks für Frontend als auch Backend selektiert und verwendet. Die einzelnen Komponenten der Anwendung werden mittels Docker \textit{containerisiert}.


\printbibliography

\end{document}
